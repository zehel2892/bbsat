
% !TEX program = pdflatex

\documentclass[9pt]{IEEEtran}
%\documentclass[12pt]{article}
\usepackage{amsmath}
\usepackage{amssymb} % for therefore implies
\usepackage{afterpage} % for leavingg blank pages
\usepackage{color}
\usepackage{listings} % for source code
\usepackage[hidelinks]{hyperref} % for creating hyperlinks
\usepackage{graphicx} % for images
\graphicspath{ {images/} } % the folder in which the images are kept

% Define the colors required for the source code typeset
\definecolor{dkgreen}{rgb}{0,0.6,0}
\definecolor{gray}{rgb}{0.5,0.5,0.5}
\definecolor{mauve}{rgb}{0.58,0,0.82}

\lstset{frame=tb,
  language=C++,
  aboveskip=3mm,
  belowskip=3mm,
  showstringspaces=false,
  columns=flexible,
  basicstyle={\small\ttfamily},
  numbers=none,
  numberstyle=\tiny\color{gray},
  keywordstyle=\color{blue},
  commentstyle=\color{dkgreen},
  stringstyle=\color{mauve},
  breaklines=true,
  breakatwhitespace=true,
  tabsize=3
}

\begin{document}
\title{\textbf{Bharath Balloon Satellite status report on 4 March 2016}}



\author{Sayan Bhattacharjee \\
				\emph{Bharath University} \\
				\emph{Aeronautical Department} \\
				\emph{2 $^{ nd}$ year BTECH stduent} \\
				\emph{Chennai 600073}}
\maketitle{}

\emph{\textbf{Abstract----}}\textbf{ This document contains the status report for the development of \emph{Bharath Balloon Satellite}
																		 	and plans of development for the future
													 }\\
\textbf{\emph{Keywords----}}\textbf{[Status report , Bharath Balloon Satellite]}
\\
%\tableofcontents
\section{INTRODUCTION}
With the permission of Dr.M Sundararaj (Head of Department of Aeronautical Engineering ,Bharath University,Chennai) the Bharath Balloon satellite
project has been started and after preliminary preparation the project is in a stage of fabrication. In this document the overview of the subsystems of the
balloon satellite and the tentative deadlines for each subsystem has been given with a few proposals for a better experiment success.

\section{SUBSYSTEMS}

\subsubsection{PAYLOAD}
	The payload will be a radiation detection kit which will contain a geiger muller tube as it's main component.
	However it will be difficult to get a valid reading using only one geiger muller tube because the flux of the radiation
	is small and the probability of an ionizing particle hitting a single geiger mmuller tube of low test section area is very low.
	The probability will continue decreasing as the balloon satellite climbs higher because the flux of the radiation particles
	is highest at the sea level.\\
	\emph{\textbf{Hence it is proposed that multiple geiger muller tubes are connected in parallel to increase the surface area of the
	radiation test section. The increase of the probability of an ionizing particle hitting the test section will not be very high
	but during a flight time of approximately 3 to 4 hours higher count of radiation detection is expected for a test section of
	higher surface area. For best results a square test section area will be required and by using a square test section area we can
	verify the radiation flux at the sea level to verify our experiment and if the experimental flux values are similar to the values
	calculated by other scientists we can state that the radiation flux calculated at different heigths will also be correct.}}
\\
\subsubsection{ONBOARD COMPUTING AND DATA HANDLING}
	The onboard computing and data handling will be managed using an \textbf{\emph{Arduino Mega 2560 R3}} as it's computing unit.
	The Arduino Mega 2560 R3 is selected because it has a total of 100 pins havinng both digital and analog types of  data pins to use.
	It has 3 Tx and 3 Rx pins which can be used for connecting the GPS and the required UHF transmitter.
	Different analog and data pins are available which will be required for different sensors that are to be used.
	\\
\subsubsection{COMMUNICATION SYSTEM AND SYNCHRONIZATION}
	The communication system will be based on UHF band for long distance radio connection and live data transfer.
	However professional help is required for this subsystem.
	The balloon satellite will only contain a simplex type connection where a single UHF transmitter will be placed on the
	satellite and only used for transmitting an un-encrypted data stream to the ground station.
	The data packets will be arranged to allow the GPS co-ordinates to be transmitted at an interval of every 5 seconds ,with the
	corresponding pressure and temperature data. If any radiation is detected by the geiger muller tubes a bit corresponding to the
	number given to that particular geiger muller tube will be transmitted along with the time of detection and the atmospheric height
	at which the detection took place.
	Dipole or patch antennas may be used.
\subsubsection{STRUCTURAL INTEGRITY AND THERMAL CONTROL}
	The structural housing of the balloon satellite will be required to be heat insulated,water proof,airtight,having high impact strength
	and lightweight. Hence the preliminary design of the structure is made up of one internal styrofoam housing and external carbon fibre/glass fibre
	housing.
	There will be a certain air gap between the external and internal housing to insulate the internal heat and prevent it from bleeding to the
	cold atmosphere. Thin styrofoam plates will be placed on the interior of the carbon fibre/glass fibre external housing to increase the insulation
	property of the structure.
	A bleed patch will be given to allow heat ventialtion into the cold atmosphere in case of overheating.


\section{TIME REQUIRED FOR PROTOTYPE CREATION}
\subsection{PAYLOAD}
Time required = 10 days where ,
\\ 3 days for initial study and procuring materials.
\\ 3 days for building test model on bread board and testing.
\\ 4 days for making computer model and validation if the circuit board is to
		be made on a PCB.

\subsection{ONBOARD COMPUTING AND DATA HANDLING}
Time required = 5 days where,
\\ 1 day for initial study and procuring materials.
\\ 2 days for writing the required computer code.
\\ 2 days for testing.
\subsection{COMMUNICATION SYSTEM AND SYNCHRONIZATION}
Time required = 25 days where,
\\ 10 days for initial study and procuring materials.
\\ 10 days for building the equipment.
\\ 5 days for testing.
\subsection{STRUCTURAL INTEGRITY AND THERMAL CONTROL}
Time required = 5 days where,
\\ 1 day for initial study and procuring materials.
\\ 2 days for building the parts.
\\ 1 day for assembly.
\\ 1 day for thermal and airtight testing.



\section{CONCLUSION}
45 days are required for the successfull completion of the project if all the materials required are available in market .
\end{document}
